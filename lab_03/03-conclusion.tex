\section*{ЗАКЛЮЧЕНИЕ}
\addcontentsline{toc}{section}{ЗАКЛЮЧЕНИЕ}

В результате выполнения лабораторной работы были изучены принципы построения счетчиков, овладены методы синтеза синхронных счетчиков, изучены способы наращивания разрядности синхронных счетчиков.

\pagebreak

\section*{Контрольные вопросы}
\addcontentsline{toc}{section}{Контрольные вопросы}

\noindent\textbf{1.} Что называется счётчиком?\newline

\noindent\textbf{Счётчик} -- это операционный узел ЭВМ, предназначенный для выполнения счёта, кодирования в определённой системе счисления и хранения числа сигналов импульсного типа, поступающих на счётный вход.
\newline

\noindent\textbf{2.} Что называется коэффициентом пересчёта? \newline

\noindent\textbf{Коэффициент пересчёта} -- число входных сигналов, которое возвращает схему в начальное состояние, в качестве которого может быть взято любое её состояние.
\newline

\noindent\textbf{3.} Перечислить основные классификационные признаки счётчиков.\newline

\noindent \textbf{По значению модуля счёта}:\newline

\noindent - Двоичные счётчики ($M = 2^n,$ $n$ - кол-во двоичных разрядов)

\noindent - Двоичное кодированные счётчики

\noindent - Счётчики с одинарным кодированием (состояние представлено местом расположения едиснвтенной единицы)\newline

\noindent Помимо этих, существуют счётчики классификации по направлению счёта, по способу организации межразрядных связей, по порядку изменения состояний и по способу управления переключением триггеров во время счёта.\newline

%\clearpage
\noindent\textbf{4.} Указать основные параметры счётчиков.\newline

\noindent- Модуль счёта $M$

\noindent- Емкость счётчика $N$

\noindent- Статические и динамические параметры счётчика (максимальная частота счёта, минимальные длительности различных импульсов).\newline

\noindent\textbf{5.} Что такое время установки кода счётчика?\newline

\noindent \textbf{Время установки кода счётчика} – один из параметров, влияющих на его быстродействие. Время установки кода $t_{set}$ равно времени между моментом поступления входного сигнала и моментом установки счетчика в новое устойчивое состояние.
\newline

\noindent\textbf{6.} Объяснить работу синхронного счётчика с параллельным переносом, оценить его быстродействие.\newline

\noindent Синхронные счётчики строятся на синхронных тригеррах, синхронизирующие входы объединены. Счётные сигналы подают на входы. Поэтому триггеры переключатся одновременно, Отсюда сделаем вывод, что время задержки распостранения сигнала от счетного входа до выходов его триггеров равно времени задержки распостранения сигнала любого триггера счетчика от $C$-входа до его выхода.\newline

\noindent Максимальная частота -- при параллельном образовании сигналов. Сигналы переноса формируется в каждом разряде, с помощью логических схем. В качестве триггеров - синхронные триггеры с динамическим управлением.\newline

\noindent В синхронном двоичном суммирующем счётчике с параллельным переносом, построенном на $JK$-триггерах, функции возбуждения формируются параллельно.\newline

\noindent\textbf{7. } Объяснить методику синтеза синхронных счётчиков на двухступенчатых $JK$- и $D$-триггерах.\newline

\noindent Синтез синхронного счетчика как цифрового автомата содержит \textbf{7 этапов:} \newline

\noindent - Определение числа триггеров счетчика, исходя из модуля счета $M$ и максимального состояния $L$ счётчика: $n1 = ]log_{2}M[, n2 = ]log_{2}L[$, где $]...[$ -- округление до ближайшего большего целого числа.

\noindent - Составление обобщенной таблицы переходов счётчика и функций возбуждения триггеров.

\noindent - Минимизация функции возбуждения триггеров счётчика.

\noindent - Перевод минимизированных функций возбуждения в заданный базис логических функций.

\noindent - Построение функциональной схемы счётчика

\noindent - Проверка полученной схемы счётчика на самовосстановление после сбоев.