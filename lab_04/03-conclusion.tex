\section*{ЗАКЛЮЧЕНИЕ}
\addcontentsline{toc}{section}{ЗАКЛЮЧЕНИЕ}

В ходе выполнения лабораторной работы были изучены принципы построения, практического
применения и экспериментального исследования мультиплексоров.

\section*{Контрольные вопросы}
\addcontentsline{toc}{section}{Контрольные вопросы}

\noindent\textbf{1.} Что такое мультиплексор?\newline

\noindent\textbf{Мультиплексор} -- это функциональный узел, имеющий $n$ адресных входов и $N=^2n$ информационных входов и выполняющий коммутацию на выход того информационного сигнала, адрес (т.е. номер) которого установлен на адресных входах. Мультиплексор переключает сигнал с одной из $N$ входных линий на один выход.
\newline

\noindent\textbf{2.} Какую логическую функцию выполняет мультиплексор? \newline

\noindent
$$ Y = EN \bigvee\limits_{j=0}^{2^n - 1} D_{j} m_{j}(A_{n-1}, A_{n - 2}, ... , A_{i}, ... , A_{1}, A_{0})$$

\noindent $A{i}$ -- адресные входы и сигналы ($i =0, 1, ... n - 1$)

\noindent $D_{j}$ -- информационные входы и сигналы ($j = 0, 1, ..., 2^n-1$)

\noindent $m_{j}$ -- конституента числу, образованному двоичным кодом сигналов на адрессных входах

\noindent $EN$ -- вход и сигнал разрешения (стробирования)\newline

\noindent\textbf{3.} Каково назначение и использование входа разрешения?\newline

\noindent Вход $EN$ используется для:\newline

\noindent -- разрешения работы мультиплексора

\noindent -- стробирования

\noindent -- наращивания числа информационных входов\newline

\noindent При $EN = 1$, разрешается работа мультплексора, при $EN$ -- работа запрещена.\newline

\noindent\textbf{4.} Какие функции может выполнять мультиплексор?\newline

\noindent Мультиплексоры широко применяются для построения:\newline

\noindent -- коммутаторов-селекторов,

\noindent -- постоянных запоминающих устройств емкостью бит

\noindent -- комбинационных схем, реализующих функции алгебры логики

\noindent -- преобразователей кодов (например, параллельного кода в последовательный) и других узлов.

\noindent\textbf{5.} Какие существуют способы наращивания мультиплексоров?\newline

\noindent Существует два способа наращивания коммутируемых каналов:\newline

\noindent -- по пирамидальной схеме соединения мультиплексоров меньшей размерности

\noindent -- путём выбора мультиплексора группы информационных входов по адресу (т.е. номеру) мультиплексора с помощью дешифратора адреса мультиплексора группы, а затем выбором информационного сигнала мультиплексором группы по адресу информационного сигнала в группе.\newline

\noindent\textbf{6.} Поясните методику синтеза формирователя ФАЛ на мультиплексоре\newline

\noindent Реализация ФАЛ $n$ переменных на мультплексоре с $n$ адресных входами тривиальнa: на адресные входы подаются переменные, на информационные входы -- значения ФАЛ на соответсвующих наборах переменных. На выходе получаем значения ФАЛ в соответсвии с наборами переменных. В этом случае мультплексор -- ПЗУ.\newline

\noindent Для реализации ФАЛ $n + 1$ переменных на адресные входы мультплексора подаются $n$ переменных, на информационных входы $n + 1$-ая переменная (или ее инверсия), константы 0 или 1 (в соответсвии со значениями ФАЛ)\newline

\noindent\textbf{7. } Почему возникают ложные сигналы на выходе мультиплексора? Как их устранить?\newline

\noindent Для исключения на выходе ложных сигналов (их вызывают гонки входных сигналов), вход $EN$ используется как стробирующий. Для выделения полезного сигнала на вход $EN$ подается сигнал в интервале времени, свободном от действия ложных сигналов.
\end{document}
