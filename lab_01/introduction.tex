\Introduction

Цель работы – изучить схемы асинхронного RS-триггера, который является
запоминающей ячейкой всех типов триггеров, синхронных RS- и D-триггеров со
статическим управлением записью и DV-триггера с динамическим управлением записью.

Для достижения поставленной цели необходимо выполнить следующие задачи:

\begin{itemize}[$\bullet$]
    \item исследовать работу асинхронного RS-триггера с инверсными входами в статическом режиме;
    \item исследовать работу синхронного RS-триггера в статическом режиме;
    \item исследовать работу синхронного D-триггера в статическом режиме;
    \item исследовать схему синхронного D-триггера с динамическим управлением записью в статическом режиме;
    \item исследовать схему синхронного DV-триггера с динамическим управлением записью в динамическом режиме;
    \item исследовать работу DV-триггера, включенного по схеме TV-триггера.
\end{itemize}